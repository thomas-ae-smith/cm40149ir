\documentclass{acm_proc_article-sp}
\usepackage{graphicx}
\usepackage{tabularx}


\begin{document}

\title{Information-Rich Cursors as a Method of Improving Embodiment in Computer-Supported Collaborative Work}
\subtitle{CM40149 Individual Report}

\numberofauthors{1} 
\author{
	\alignauthor Thomas Smith\\
       \affaddr{Centre for Digital Entertainment}\\
       \email{taes22@bath.ac.uk}
}
\date{03 May 2014}


\maketitle
\begin{abstract} %81 words  %no more than 150 words
This paper provides an investigation into the literature surrounding the role of cursors, carets and similar gaze indicators in enabling effective remote collaboration via computer-supported collaborative work (CSCW) groupware systems. Specifically, it evaluates the claims made in previous work on the topic, and attempts to expand upon the recommendations provided for groupware designers when supporting effective remote collaboration.
The investigation and review are presented within the context of a brief investigation into the cursor-like functionality provided by two contemporary groupware systems.
\end{abstract}

% A category with the (minimum) three required fields
% \category{H.5.3}{Information Interfaces and Presentation}{Computer supported cooperative work}
%A category including the fourth, optional field follows...
% \category{D.2.8}{Software Engineering}{Metrics}[complexity measures, performance measures]
 % \terms{Design, Experimentation, Human Factors}

\keywords{Groupware, HCI, CSCW} % NOT required for Proceedings

\section{Introduction}

In a collocated group task, there is often a high degree of contextual awareness about the location and focus of the other collaborators available \cite{olsondistance2000}. This is typically not apparent during remote collaboration, unless specific effort is made by the platform to communicate aspects of it via some form of virtual embodiment within the system. For some systems and tasks, this embodiment may be simple --- e.g. a list of online users denoting the presence or absence of users within a session. In other approaches, particularly systems providing a shared workspace, the virtual embodiment may be augmented with additional task-relevant information.

\subsection{Virtual Embodiment}
As the task and nature of the shared environment varies, so too does the nature of the virtual embodiment approach typically chosen, and the terminology used to refer to it. Typically, a caret is the point in a text-based workspace at which changes would be applied, though this can be stretched to become a selection region. That concept also extends to image-based workspaces, where it is occasionally called a marquee. A cursor is the locus of the user's interaction with their own system, used typically to interact with the operating system and perform other meta-activities such as setting or changing the format of content at the caret. The cursor location would provide comparatively little intentional information to collaborators, and so is less frequently shared than the caret --- except in shared workspaces that involve locating tasks, where it is used as a pointing tool.

A user's avatar situated in a virtual environment is that user's embodiment within the system. Interaction is often limited to the parts of the virtual environment immediately visible to or adjacent to the user's avatar, and so it can function similar to a caret --- indicating \emph{where} remote changes are likely to occur. Avatars are capable of expressing richer information about the abilities or intent of a user: for example, in many systems the direction an avatar is facing indicates what may be visible to that user. In some approaches, an explicit `gaze indicator' is used to denote the object currently selected by a user, making the caret-like indications even more explicit.

\subsection{Embodiment Augmentation}
Each of these forms of virtual embodiment provides additional information that can help to improve the effectiveness of groupware systems, by indicating the probable locations of future edits by another user. Further details can be provided in order to enable differentiation between remote users, for example by colour-coding carets to match the list of online users, or allowing individual avatar personalisations. These allow identification of particular users, which can help to guide predictions of future actions. While a simple level of embodiment provides benefits related to basic awareness of other users, \emph{information-rich} embodiment can begin to approach the level of contextual awareness of the users available in collocated collaboration tasks.

% Both google and second life provide different forms of information-rich embodiment

\subsection{Document Overview}
An initial context for the discussion will be provided via an overview of the investigation and recommendations presented by Wrigglesworth et. al. in \cite{wrigglesworth2014design}. With reference to a number of other studies in related areas, the claim that additional gaze information would be the most beneficial improvement to the presented groupware systems' embodiment implementations is investigated and refined. Observations from the surrounding literature are used to suggest a more complete recommendation for groupware designers, which is presented in the final conclusion.

\section{Background}

In Wrigglesworth et. al. \cite{wrigglesworth2014design}, a number of guidelines are presented relating to design factors for groupware systems that influence the effectiveness of support for remote collaborative work. Specifically, it is suggested that gaze information is the major augmentation lacking from existing embodiment implementations, in light of the outcomes of a two-part investigation into the features supporting collaboration provided by contemporary systems

\subsection{Approach}
An empirical investigation was carried out into the features supporting collaborative work within two differing groupware systems: Second Life, a 3D virtual world; and Google Docs, a collaborative text-based document editor. In each system an appropriate task was undertaken, and observations are provided about the aspects of each system that aided or hindered completion of the task.

In Second Life, a spatial organisation task was performed relying on incomplete shared map information. It was necessary to communicate sector information and collaboratively assemble a complete picture of the full map, and then arrange a collection of virtual objects to represent this information. There was potential for collision between task members if simultaneous attempts were made to reorganise portions of the arrangement.

In Google Docs, an ideation task was undertaken, followed by the creation of a short fictional passage. The initial idea generation stages were fairly loosely coupled, but there was a necessary fact-checking and critical review process that led to a number of edits being made. The fiction writing was a particularly difficult task to perform collaboratively, and throughout the process there was potential for edits made to conflict with other information in the document, if the user was not aware that changes had been made.

% \section{Results}

\section{Related Work}

Actual gaze does not typically match cursor movement, even in the case of free-moving `pointer' type cursors \cite{byrne1999eye}. Talk about cursors as an indicator of potential interaction location - if an object or a location is selected then a user may be about to change it. However, their actual gaze may be elsewhere --- they may be about to select another location and apply changes there instead. Communication of gaze information may help to improve awareness of this possibility --- if gaze and caret are not collocated. There may be privacy concerns --- users are aware that if another user's caret is not currently active, they may be reviewing content elsewhere in the workspace, or they may be distracted by another, external task. It may not be desirable to make these instances of task focus-loss apparent and explicit.

Mori et. al. \cite{mori2011collaborative} describe improvements that may be made to the Google Docs user interface, including to the embodiment implementation, that may increase the accessibility for assistive interfaces. It is apparent that some of the changes may also improve the awareness of all users

In \cite{andreas2010fostering}, Andreas et. al describe some of the issues with the traditional avatar representation, and provide a number of descriptions of augmentations they used to improve the environment's suitability for a collaborative task --- specifically, learning.


\section{Discussion}
Talk about the difference between actual gaze, communicated gaze, and cursor/caret/avatar location.

Paragraph on status indicators (typing/busy/away).

Paragraph on task relevance --- discuss coupling and collision potential: environment type, task type, roles. (Influence on cursor-related awareness on role formation? `She's working on that, I'll do this'?)

\section{Conclusion}

The existing literature on the relation between gaze and intention, and on the effectiveness of shared pointers for improving awareness, indicates that we should revise and extend the claims in the following manner: \$outcome. 

Also note that an unrelated claim --- about the desirability of voice/video communication channels, is supported by the paper on search and rescue.


%There are certain properties of the physical world that support communication between individuals and groups such as the visibility of activities and presence of participants. In digital systems these properties are not always apparent. Consequently, participants are unable to react to social cues and structure their behaviour to the social context  -- which can result in unsatisfactory interactions \cite{ericksonsocial2000}. Socially translucent systems aim to provide more social information so that participants are more aware of the interactions and activities taking place within a digital space. This awareness can help make collaborative systems operate more effectively by affording more coherence to group activities \cite{dinginforming2011}.

% A number of aspects of the two groupware systems we used were relevant to the effectiveness of the tasks we undertook -- as discussed below. The difference in nature of the task affects the functionality required in order to best support successful completion \cite{straus1999testing} -- and not all helpful features of each platform were discovered during the course of each task.


% \subsection{Gaze}

% Gaze is an important part of human communication. Collocated interlocutors can use the social cues afforded by gaze to read each others behaviour as well as signal their own. This is particularly relevant in terms of focus of attention, maintaining roles, turn-taking and signalling attitudes \cite{argyle1976gaze}.

% Second Life and Google Docs are both unable communicate the gaze of participants, however, they are able to provide features that help compensate for the loss of it. Second Life shows a line of sight from a avatar's head to the object of their focus. This allows for participants to know that a teammate is active in the task and which object they are actively working on -- however this information may become `stale' as it is only updated by explicit interaction.

% Google Docs visualises the attention of participants by assigning each one a coloured cursor. This gives an indication of what they are currently working on and therefore where their attention is focused. Without this visual cue it would be very difficult to tell who was contributing to the different parts of the document.

% The coloured cursor would not always be a completely reliable indicator and could be misleading at times because a participant could leave their cursor at one position while they read a different part of the document -- as with the Second Life gaze indicator, above. The document could span multiple pages and so it was not clear which page or section a user was viewing. A solution to this problem could be to give some sort of indication of what part of each contributor is looking at; maybe a coloured frame indicating their current view space.



% \subsection{Roles}

% People need to cooperate in collaborative work. Because of differing experience, personality, preferences and ability, different roles will emerge, even they are not agreed in advance \cite{barlow2013emergent}. Roles are good for organizing the group work, and appropriate roles assignment or development can improve the work efficiency.
 
% In the Second Life experience, a special role emerged. One team member became responsible for controlling the arrangement of blocks. This action increased the speed of the process, and prevents the possibility that two players might attempt to move one block at the same time. This responsibility was not given in advance -- it naturally formed during the experiment. This could cause trouble for bigger groups, as some participants may ignore the existing role. The reason why this role emerged during the process was not only because there were only three people in the game, but also due to differing prior experience with the games rules. So we cannot anticipate what kind of role will be helpful in this situation. According to this discussion, familiarity with the work environment and prior knowledge of appropriate roles will be useful for collaborative work.






\bibliographystyle{abbrv}
\bibliography{cursors}

\balancecolumns
\end{document}
